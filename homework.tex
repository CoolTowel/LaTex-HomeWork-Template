\documentclass[a4paper]{article}
%\usepackage[scheme = plain]{ctex} %调用 xeCJK 宏包,启用中文支持
%   \setCJKmainfont{SimSun} %设置 CJK 主字体为 SimSun (宋体)


\usepackage{float}
\usepackage{amsmath}    %数学宏包
\usepackage{amstext}    %在数学公式中插入少量文本
\usepackage{amsthm}    %排版定理和证明
\usepackage{amsfonts}
\usepackage{siunitx}    %国际单位制宏包
\usepackage{geometry}    %页边距
\usepackage{titling}    %标题自定义
\usepackage{fancyhdr}    %页眉页脚
%\usepackage{microtype}  %调整局部字间距,不兼容 xelatex

%文档基本样式
\oddsidemargin=0.35in
\textwidth=6in
\textheight=9in
\setlength{\parskip}{1em}
%文档基本样式

%作业相关信息
\newcommand{\HomeworkTitle}{Homework \#1}
\newcommand{\DueDate}{Mar 27, 2018}
\newcommand{\Class}{High Energy Astrophysics}
\newcommand{\ClassInstructor}{Professor Edwin Hubble}
\newcommand{\AuthorName}{Xiaoming Li}
\newcommand{\AuthorScholar}{110159753951}
%作业相关信息

%标题页
\title{
    \vspace{1.5in}
    \LARGE{\textbf{\Class: \HomeworkTitle}}\\
    \normalsize\vspace{0.1in}\large{Due on \DueDate}\\
    \vspace{0.08in}\large{\textsc{\ClassInstructor}}
    \vspace{4.5in}
    }

\author{
    \Large{\textsc{\AuthorName}}\\
    \vspace{0.2in}\large{\textrm{\AuthorScholar}}
    }

\date{}
%标题页

%页眉页脚
\pagestyle{fancy}
\chead{\Class: \HomeworkTitle}
\cfoot{\thepage}
\renewcommand{\headrulewidth}{0.4pt}
\renewcommand{\footrulewidth}{0.4pt}
%页眉页脚

%自定义
\newcommand{\problem}[1]{\addtolength\parskip{3em}\noindent\textbf{\Large{#1}}\addtolength\parskip{-3em}}
\newcommand{\solution}{\noindent\textsl{\large{Solution:}}}

%数学符号
\newcommand*{\dif}{\mathop{}\!\mathrm{d}}

\begin{document}


\begin{titlepage}
\maketitle
\thispagestyle{empty}
\end{titlepage}

\problem{1. Special relativity}

\solution

The muons in this problem produced at a height of $H=\SI{6}{\kilo\meter}$, after $t_a=\frac{H}{v}$ arrived Earth's surface. It means the lifetime of them should longer than $t_a$ :
\[
t_{\textrm{life}}\geq t_a
\]

According to the theory of special relativity, this particles should have a longer lifetime if they are measured by the observer which on the Earth's surface. And their lifetime are given by the Lorentz transformation :
\[
t_{\textrm{life}}=\gamma  t_{\textrm{rest}} =\frac{t_{\textrm{rest}}}{\sqrt{1-\frac{v^2}{c^2}}}
\]
where $v$ is the relative velocity between frames, $c$ is the speed of light, and $\gamma$ is the Lorentz factor.

For a relativistic particle, the relationship of its mass and Lorentz factor is:
\[
E=\gamma mc^2
\]
where $m$ is the invariant mass.

By the above formula we have:
\[
\gamma\geq\sqrt{\frac{H^2}{c^2t_{\textrm{rest}}^2}+1}
\]

In this problem, $mc^2=\SI{105}{\mega\electronvolt}$, $t_{\textrm{rest}}=\SI{2200}{\nano\second}$. Hence the lower limit of the total energy of the muons in this case is:
\[\begin{aligned}
E_{\textrm{lower limit}}&=\gamma mc^2 \\
&= m c^2 \sqrt{\frac{H^2}{c^2t_{\textrm{rest}}^2} + 1} \\
& \approx \SI{105}{\mega\electronvolt} \times 9.15 \\
& \approx \SI{961}{\mega\electronvolt}
\end{aligned}
\]


\problem{2. Blackbody radiation}

Example Example Example Example Example 

\end{document} 